%
%  $Description: Author guidelines and sample document in LaTeX 2.09$ 
%
%  $Author: ienne $
%  $Date: 1995/09/15 15:20:59 $
%  $Revision: 1.4 $
%

\documentclass[times, 10pt,twocolumn]{article} 
\usepackage{latex8}
\usepackage{times}

%\documentstyle[times,art10,twocolumn,latex8]{article}

%------------------------------------------------------------------------- 
% take the % away on next line to produce the final camera-ready version 
\pagestyle{empty}

%------------------------------------------------------------------------- 
\begin{document}

\title{DADTKV}

\author{Lucas Pinto\\
Instituto Superio Técnico\\
Universidade de Lisboa\\
lucas.f.pinto@tecnico.ulisboa.pt\\
% For a paper whose authors are all at the same institution, 
% omit the following lines up until the closing ``}''.
% Additional authors and addresses can be added with ``\and'', 
% just like the second author.
\and
Second Author\\
Institution2\\
First line of institution2 address\\ Second line of institution2 address\\ 
SecondAuthor@institution2.com\\
}

\maketitle
\thispagestyle{empty}

\begin{abstract}
   This project focuses on developing \textbf{DADTKV}, a distributed transactional key-value store.
   \textbf{DADTKV} enables concurrent data access through a multi-tier architecture.
   Clients submit transactions using a specialized library, and transactions are coordinated
   through leases and consensus algorithms. This project leverages C\# and gRPC to create
   a robust and fault-tolerant distributed system for efficient data management.
\end{abstract}



%------------------------------------------------------------------------- 
\Section{Introduction}

This project introduces a complex architecture with three tiers, encompassing client applications,
transaction managers, and lease manager servers. \textbf{DADTKV}'s primary data unit is the \textit{DadInt}
key-value pair, with a focus on strict serializability to ensure data consistency.

%------------------------------------------------------------------------- 
\Section{Implementations}

\textbf{TODO}

%------------------------------------------------------------------------- 
\SubSection{Manager}

The Manager process is the main entry point for the system. It is responsible for reading
the system configuration and starting the other processes.

Lease Managers are the first processes to be created and then the Transaction Managers.
This, however, does not guarantee that the Lease Managers are ready before the Transaction Managers.
Because of that, each process notifies the Manager, through a \textit{gRPC} call, that it is ready to work.

After all Lease and Transaction Managers are ready, the Manager demands each one to start
their operations. It is also at this time that the Manager will start the Client processes.

%------------------------------------------------------------------------- 
\SubSection{Clients}

Client processes are very simple, in the state that they cycle through the list of operations
that they have to execute, over and over again.

They will just send requests to a Transaction Managers predetermined by the Manager
and wait for the response.

In case the given Transaction Manager is suspected of not being available,
the Client will try another Transaction Manager it thinks is correct.
This works by detecting if the gRPC call fails because the server is unable to accept it.

%------------------------------------------------------------------------- 
\SubSection{Transaction Managers}

\textbf{TODO}

%------------------------------------------------------------------------- 
\SubSection{Lease Managers}
When the Lease Manager is ordered to start by the Manager, it will create the
timer responsible for the creation of \textit{Paxos} instances.
This timer runs every \textit{slotDuration} milliseconds.

Lease requests are stored in a \textit{buffer queue}, so that no requests are lost between
between \textit{Paxos} instances. Whenever a new instance is created, it will lock
and fetch all requests from the \textit{buffer}, clear, and unlock it for more
requests to be received.

\textit{Paxos} is extended to allow the Proposer to provide a \textit{SHA256 hash} of
the Lease requests (value) it wishes to propose on the Prepare message. The Propose
message comes as a response to the \textit{gRPC} Prepare call, and will include the
respondents' Lease requests if the hashes are different -- this allows for the Proposer
to make more complete proposal.

%------------------------------------------------------------------------- 
\SubSection{Type-style and fonts}

Wherever Times is specified, Times Roman may also be used. If neither is 
available on your word processor, please use the font closest in 
appearance to Times that you have access to.

MAIN TITLE. Center the title 1-3/8 inches (3.49 cm) from the top edge of 
the first page. The title should be in Times 14-point, boldface type. 
Capitalize the first letter of nouns, pronouns, verbs, adjectives, and 
adverbs; do not capitalize articles, coordinate conjunctions, or 
prepositions (unless the title begins with such a word). Leave two blank 
lines after the title.

AUTHOR NAME(s) and AFFILIATION(s) are to be centered beneath the title 
and printed in Times 12-point, non-boldface type. This information is to 
be followed by two blank lines.

The ABSTRACT and MAIN TEXT are to be in a two-column format. 

MAIN TEXT. Type main text in 10-point Times, single-spaced. Do NOT use 
double-spacing. All paragraphs should be indented 1 pica (approx. 1/6 
inch or 0.422 cm). Make sure your text is fully justified---that is, 
flush left and flush right. Please do not place any additional blank 
lines between paragraphs. Figure and table captions should be 10-point 
Helvetica boldface type as in
\begin{figure}[h]
   \caption{Example of caption.}
\end{figure}

\noindent Long captions should be set as in 
\begin{figure}[h] 
   \caption{Example of long caption requiring more than one line. It is 
     not typed centered but aligned on both sides and indented with an 
     additional margin on both sides of 1~pica.}
\end{figure}

\noindent Callouts should be 9-point Helvetica, non-boldface type. 
Initially capitalize only the first word of section titles and first-, 
second-, and third-order headings.

FIRST-ORDER HEADINGS. (For example, {\large \bf 1. Introduction}) 
should be Times 12-point boldface, initially capitalized, flush left, 
with one blank line before, and one blank line after.

SECOND-ORDER HEADINGS. (For example, {\elvbf 1.1. Database elements}) 
should be Times 11-point boldface, initially capitalized, flush left, 
with one blank line before, and one after. If you require a third-order 
heading (we discourage it), use 10-point Times, boldface, initially 
capitalized, flush left, preceded by one blank line, followed by a period 
and your text on the same line.

%------------------------------------------------------------------------- 
\SubSection{Footnotes}

Please use footnotes sparingly%
\footnote
   {%
     Or, better still, try to avoid footnotes altogether.  To help your 
     readers, avoid using footnotes altogether and include necessary 
     peripheral observations in the text (within parentheses, if you 
     prefer, as in this sentence).
   }
and place them at the bottom of the column on the page on which they are 
referenced. Use Times 8-point type, single-spaced.


%------------------------------------------------------------------------- 
\SubSection{References}

List and number all bibliographical references in 9-point Times, 
single-spaced, at the end of your paper. When referenced in the text, 
enclose the citation number in square brackets, for example~\cite{ex1}. 
Where appropriate, include the name(s) of editors of referenced books.

%------------------------------------------------------------------------- 
\SubSection{Illustrations, graphs, and photographs}

All graphics should be centered. Your artwork must be in place in the 
article (preferably printed as part of the text rather than pasted up). 
If you are using photographs and are able to have halftones made at a 
print shop, use a 100- or 110-line screen. If you must use plain photos, 
they must be pasted onto your manuscript. Use rubber cement to affix the 
images in place. Black and white, clear, glossy-finish photos are 
preferable to color. Supply the best quality photographs and 
illustrations possible. Penciled lines and very fine lines do not 
reproduce well. Remember, the quality of the book cannot be better than 
the originals provided. Do NOT use tape on your pages!

%------------------------------------------------------------------------- 
\SubSection{Color}

The use of color on interior pages (that is, pages other
than the cover) is prohibitively expensive. We publish interior pages in 
color only when it is specifically requested and budgeted for by the 
conference organizers. DO NOT SUBMIT COLOR IMAGES IN YOUR 
PAPERS UNLESS SPECIFICALLY INSTRUCTED TO DO SO.

%------------------------------------------------------------------------- 
\SubSection{Symbols}

If your word processor or typewriter cannot produce Greek letters, 
mathematical symbols, or other graphical elements, please use 
pressure-sensitive (self-adhesive) rub-on symbols or letters (available 
in most stationery stores, art stores, or graphics shops).

%------------------------------------------------------------------------ 
\SubSection{Copyright forms}

You must include your signed IEEE copyright release form when you submit 
your finished paper. We MUST have this form before your paper can be 
published in the proceedings.

%------------------------------------------------------------------------- 
\SubSection{Conclusions}

Please direct any questions to the production editor in charge of these 
proceedings at the IEEE Computer Society Press: Phone (714) 821-8380, or 
Fax (714) 761-1784.

%------------------------------------------------------------------------- 
\nocite{ex1,ex2}
\bibliographystyle{latex8}
\bibliography{latex8}

\end{document}

